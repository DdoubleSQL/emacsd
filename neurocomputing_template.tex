%% This is file `elsarticle-template-1-num.tex", %%

%% copyright 2009 Elsevier Ltd %%

%% This file is part of the "Elsarticle Bundle". %% --------------------------------------------- %%

%% It may be distributed under the conditions of the LaTeX Project Public %% License, either version 1.2 of this license or (at your option) any %% later version. The latest version of this license is in %% /svn/elsbst/trunk/elsarticle-template-1-num.tex $ %%

\documentclass[final,5p,times]{elsarticle}

%% Use the option review to obtain double line spacing %% \documentclass[preprint,review,12pt]{elsarticle}

%% Use the options 1p,twocolumn; 3p; 3p,twocolumn; 5p; or 5p,twocolumn %% for a journal layout:

%% \documentclass[final,1p,times]{elsarticle}

%% \documentclass[final,1p,times,twocolumn]{elsarticle} %% \documentclass[final,3p,times]{elsarticle}

%% \documentclass[final,3p,times,twocolumn]{elsarticle} %% \documentclass[final,5p,times]{elsarticle}

%% \documentclass[final,5p,times,twocolumn]{elsarticle}

%% if you use PostScript figures in your article

%% use the graphics package for simple commands %% \%usepackage{graphics}

%% or use the graphicx package for more complicated commands %% \%usepackage{graphicx}

%% or use the epsfig package if you prefer to use the old commands %% \%usepackage{epsfig}

%% The amssymb package provides various useful mathematical symbols \%usepackage{amssymb}

\usepackage[fleqn]{amsmath}

%% The amsthm package provides extended theorem environments %% \%usepackage{amsthm}

%% The lineno packages adds line numbers. Start line numbering with %% \begin{linenumbers}, end it with \end{linenumbers}. or switch it on %% for the whole article with \linenumbers after \end{frontmatter}. %% \%usepackage{lineno}

%% natbib.sty is loaded by default. However, natbib options can be %% provided with \biboptions{...} command. Following options are %% valid:

%% round - round parentheses are used (default) %% square - square brackets are used [option] %% curly - curly braces are used {option} %% angle - angle brackets are used

%% semicolon - multiple citations separated by semi-colon %% colon - same as semicolon, an earlier confusion %% comma - separated by comma %% numbers- selects numerical citations

%% super - numerical citations as superscripts

%% sort - sorts multiple citations according to order in ref. list

%% sort&compress - like sort, but also compresses numerical citations %% compress - compresses without sorting %%

%% \biboptions{comma,round}

% \biboptions{}

\journal{neurocomputing}

\begin{document}

\begin{frontmatter}

%% Title, authors and addresses

%% use the tnoteref command within \title for footnotes; %% use the tnotetext command for the associated footnote;

%% use the fnref command within \author or \address for footnotes; %% use the fntext command for the associated footnote;

%% use the corref command within \author for corresponding author footnotes; %% use the cortext command for the associated footnote; %% use the ead command for the email address, %% and the form \ead[url] for the home page: %%

%% \title{Title\tnoteref{label1}} %% \tnotetext[label1]{}

%% \author{name\corref{cor1}\fnref{label2}} %% \ead{email address} %% \ead[url]{home page} %% \fntext[label2]{} %% \cortext[cor1]{}

%% \address{Address\fnref{label3}} %% \fntext[label3]{}

\title{Title\tnoteref{label1}}

%\tnotetext[label1]{corresponding author} \author{A\fnref{label2}}

\author{Bcorref{cor1}\fnref{label3}} %\ead{}
\author{c$^3$\corref{}} %\ead[url]{home page}

\fntext[label2]{A is a postgraduate student } 
\fntext[label3]{B} 
\fntext[label4]{c is a professor }

\cortext[cor1]{corresponding author} 
\address{$^1$ } 
\address{$^2$} 
\address{$^3$ }


\title{Biao ti}

\begin{abstract} %% Text of abstract In this paper, 

sss
\end{abstract}

\begin{keyword}

%% keywords here, in the form: keyword \sep keyword

recurrent neural networks\sep finite-time stability \sep tunable

activation function \sep quadratic programming

%% MSc codes here, in the form: \MSc code \sep code %% or \MSc[2008] code \sep code (2000 is the default)

\end{keyword}

\end{frontmatter} %%

%% Start line numbering here if you want %%

% \linenumbers

%% main text

\section{Introduction}

\section{A}

\section{B}

\section{c}

\section{D }

\section{numerical simulations and application}

\section{conclusion} In the paper,

\section{Acknowledgment}

This work was supported by the national Science Foundation of china …..

%% The Appendices part is started with the command \appendix; %% appendix sections are then done as normal sections %% \appendix

%% \section{} %% \label{}

%% References %%

%% Following citation commands can be used in the body text: %% Usage of \cite is as follows:

%% \cite{key} ==>> [#]

%% \cite[chap. 2]{key} ==>> [#, chap. 2] %% \citet{key} ==>> Author [#]

%% References with bibTeX database: %\section*{References}

\bibliographystyle{mode1-num-names} \bibliography{}

%% Authors are advised to submit their bibtex database files. They are %% requested to list a bibtex style file in the manuscript if they do %% not want to use model1-num-names.bst.

%% References without bibTeX database:

\begin{thebibliography}{00}

\bibitem[\protect\citeauthoryear{Smith}{1999}]{A1999}

K.A. Smith, {\it neural networks for combinatorial optimization: a review of more than a decade of research.}

Informs Journal computing, 11 (1999) 15-34.

\bibitem[\protect\citeauthoryear{Li, Lou and Liu}{2012}]{LB2012}

S. Li, Y. Lou and B. Liu, {\it Bluetooth aided mobile phone localization: a nonlinear neural circuit approach.}

AcM Trans Embedded comput Syst, January 2013.

\bibitem[\protect\citeauthoryear{Lin, Li and Liu}{2012}]{B2012}

S. Lin, Y. Li and B. Liu, {\it Model-free control of Lorenz chaos using an approximate optimal control strategy.}

commun nonlinear Sci numer Simul, 12(7) (2012) 4891-4900.

\end{thebibliography}

\end{document} %%

%% End of file `elsarticle-template-1-num.tex".
